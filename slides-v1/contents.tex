\mode*

\begin{frame}
  \tableofcontents
\end{frame}

\section{Vad är matematik?}

% Eleverna ska tycka att programmering är intressant och förstå att 
% mattekunskaper är bra att ha :D

\begin{frame}
  \begin{example}<1,3>
    \begin{itemize}
      \item En triangel har basen \SI{\alert<3>2}{\centi\metre} och höjden 
        \SI{\alert<3>3}{\centi\metre}.
      \item En rektangel har basen \SI{\alert<3>2}{\centi\metre} och höjden 
        \SI{\alert<3>3}{\centi\metre}.
      \item Hur mycket större area har rektangeln?
    \end{itemize}
  \end{example}

  \begin{example}<2,3>
    \begin{itemize}
      \item En triangel har basen \(\SI{\alert<3>b}{\centi\metre}\) och höjden 
        \(\SI{\alert<3>h}{\centi\metre}\).
      \item En rektangel har basen \(\SI{\alert<3>b}{\centi\metre}\) och höjden 
        \(\SI{\alert<3>h}{\centi\metre}\).
      \item Kommer rektangeln alltid att ha dubbla arean?
    \end{itemize}
  \end{example}
\end{frame}

\begin{frame}
  \begin{solution}
    \begin{itemize}
      \item Ja, rektangeln kommer alltid att ha dubbla arean.
      \item Arean för rektangeln är \(bh\), medan arean för triangeln är 
        \(\frac{1}{2}bh\).
      \item \(b\) och \(h\) var ju desamma i båda fallen.
    \end{itemize}
  \end{solution}
\end{frame}

\begin{frame}
  \begin{question}
    \begin{itemize}
      \item Varför beräknas triangelns area med formeln \(\frac{bh}{2}\)?
      \item Varför beräknas rektangelns area med formeln \(bh\)?
    \end{itemize}
  \end{question}
\end{frame}

\begin{frame}
  \begin{solution}
    \begin{itemize}
      \item Det följer av hur vi definierar area.
      \item Meter för längd.
      \item Kvadratmeter för area.
    \end{itemize}
  \end{solution}

%  \pause
%
%  \begin{remark}
%    \begin{itemize}
%      \item Vi kräver även några fler grundläggande antaganden.
%      \item Om två linjer korsar varandra, då är vinkelsumman vid korsningen 
%        alltid 360 grader.
%      \item Alla rätvinkliga vinklar är 90 grader.
%    \end{itemize}
%  \end{remark}
\end{frame}

\subsection{Varför är det centralt för datasäkerhet?}

\begin{frame}
  \begin{remark}
    \begin{itemize}
      \item Vi kan inte testa säkerheten.
      \item Vi vet inte hur angriparen kommer att göra.
      \item Vi måste veta att hur angriparen än gör, så lyckas hen inte.
    \end{itemize}
  \end{remark}

  \pause

  \begin{example}
    \begin{itemize}
      \item Att testa säkerheten är som att testa att rektangeln har mer area 
        än triangeln.
    \end{itemize}
  \end{example}
\end{frame}

\begin{frame}
  \begin{solution}
    \begin{itemize}
      \item Vi behöver arbeta systematiskt och deduktivt med matematik.
    \end{itemize}
  \end{solution}
\end{frame}

\begin{frame}
  \begin{exercise}
    \begin{itemize}
      \item Anna valde sitt lösenord på följande sätt:
      \item Hon valde ett ord hon tycker om, sedan lade hon till två siffror 
        och ett specialtecken i slutet.
      \item En dator kan gissa mellan 100\,000 och 1\,000\,000 lösenord per 
        sekund.
      \item Hur lång tid tar det att knäcka Annas lösenord?
    \end{itemize}
  \end{exercise}

  \pause

  \begin{exercise}
    \begin{itemize}
      \item Lars valde sitt lösenord annorlunda.
      \item Han valde ett ord och översatte det till 1337-speak.
      \item Hur lång tid tar det att knäcka hans lösenord?
    \end{itemize}
  \end{exercise}
\end{frame}

\begin{frame}
  \begin{solution}
    \begin{itemize}
      \item<+-> Hur många ord har de att välja mellan?
      \item<+-> Svenska Akademiens Ordlista har ungefär 125\,000 ord: 
        \(125\,000^1\) möjligheter.
      \item<+-> Anna valde två siffror och ett specialtecken slumpmässigt: 
        \(10^3\) möjligheter.
      \item<+-> Lars ändrade till 1337-speak, detta är deterministiskt: \(1\) 
        möjlighet.
      \item<+-> Den \alert{långsamma} datorn behöver \(\frac{125\,000\cdot 
        10^3}{100\,000} = 1250\) sekunder, eller strax under 21 minuter för att 
        knäcka Annas lösenord.
      \item<+-> Den \alert{långsamma} datorn behöver \(\frac{125\,000\cdot 
        1}{100\,000} = 
        1.25\) sekunder för Lars lösenord.
    \end{itemize}
  \end{solution}
\end{frame}

\begin{frame}
  \begin{exercise}
    \begin{itemize}
      \item Evin valde fyra slumpmässigt valda ord ur Svenska Akademiens 
        Ordlista som sitt lösenord.
      \item Inga siffror, inga specialtecken, inget 1337-speak.
      \item Hur lång tid tar det att knäcka hennes lösenord?
    \end{itemize}
  \end{exercise}
\end{frame}

\begin{frame}
  \begin{solution}
    \begin{itemize}
      \item Evin valde fyra slumpmässiga ord: det ger \(125\,000^3\) 
        möjligheter.
      \item Den \alert{snabba} datorn behöver \(\frac{125\,000^3}{1\,000\,000} 
        = 1953125000\) sekunder, eller 62 år.
    \end{itemize}
  \end{solution}
\end{frame}

\begin{frame}
  \begin{question}
    \begin{itemize}
      \item Kommer Evins sätt att välja lösenord alltid att vara bättre?
      \item Finns det bättre gissningsalgoritmer som knäcker hennes lösenord 
        snabbare?
    \end{itemize}
  \end{question}
\end{frame}

\begin{frame}
  \begin{question}
    \begin{itemize}
      \item Egentligen borde det ta hälfen så lång tid i medel.
      \item Varför?
    \end{itemize}
  \end{question}
\end{frame}


\begin{frame}
  \begin{example}[Lösenord]
    \begin{itemize}
      \item Ovan diskuterade vi olika algoritmer för att skapa och gissa 
        lösenord.
      \item Vi kan skriva ett pythonprogram som faktiskt gissar lösenord.
    \end{itemize}
  \end{example}
\end{frame}



