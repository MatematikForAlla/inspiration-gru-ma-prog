% What's the problem?
% Why is it a problem? Research gap left by other approaches?
% Why is it important? Why care?
% What's the approach? How to solve the problem?
% What's the findings? How was it evaluated, what are the results, limitations, 
% what remains to be done?

I den här modulen skriver vi ett frågesportprogram.
Användaren får frågor som hen försöker att svara rätt på.

Syftet är att illustrera enkla algoritmiska problem och ge återanvändbara block 
som enkelt kan modifieras.

\emph{Lärandemål:}
Efter att ha genomfört modulen ska den lärande kunna:
\theorynote{%
  Genomgående kommer vi att kommentera texten utifrån ett lärandeteoretiskt 
  perspektiv och koppla tillbaka till dessa lärandemål.

  Den teoretiska utgångspunkten är variationsteori, se exempelvis
  \citetitle{NecessaryConditionsOfLearning}
  av \citeauthor{NecessaryConditionsOfLearning}
  \cite{NecessaryConditionsOfLearning}.
  För en kort genomgång, se materialet på
  \begin{quote}
    {https://github.com/dbosk/necessary-conditions-of-learning/releases/tag/20240228}
  \end{quote}
  Gå även igenom filen \texttt{slides.pdf} som finns under 
  \foreignblockquote{british}{Assets}.
}
\begin{itemize}
  \item läsa grundläggande programkod med enkla konstruktioner
    och, mestadels korrekt, förklara vad koden gör;
  \item konstruera enkla algoritmer (inte syntaktiskt korrekt programkod) på en 
    detaljnivå lämpad för programmering, exempelvis att man måste läsa bokstav 
    för bokstav och göra någonting för varje.
  \item använda existerande programkod som block i egna program, dvs 
    \enquote{remixa} existerande program till nya.
\end{itemize}
